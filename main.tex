\documentclass[twocolumn,9pt]{jsproceedings}
\RequirePackage[l2tabu,orthodox]{nag}  % 古いコマンドやパッケージを使用した場合に警告する
\usepackage[all,warning]{onlyamsmath}  % amsmath が提供しない数式環境を使用した場合に警告する
% \usepackage{flushend}  % 最終ページの2カラムの左右の高さを揃える
\usepackage{here} %図の場所の指定で[h](ここに貼る)を指定するためのパッケージ
\usepackage[dvipdfmx]{graphicx} %dvipdfmxはjpgやpngの張り込みのために使用


% タイトル
\title{小型自律移動ロボットを用いた\\つくばチャレンジ2023での取り組み}

\author{○池邉 龍宏\authorrefmark{2}、内田 璃空\authorrefmark{2}、畑中 優一郎\authorrefmark{1}、臼井 温希\authorrefmark{1}、庄司 史門\authorrefmark{1}、松井 大和\authorrefmark{1}、\\
山崎 政光\authorrefmark{1}、登内 リオン\authorrefmark{1}、林原 靖男\authorrefmark{1}、上田 隆一\authorrefmark{1}}

\etitle{Participation report in the Tsukuba Challenge 2022\\using a small mobile robot with a Raspberry Pi}

\eauthor{○Tatsuhiro IKEBE\eauthorrefmark{2}、Riku UCHIDA\eauthorrefmark{2}、Yuichiro HATANAKA\eauthorrefmark{1}、Atsuki USUI\eauthorrefmark{1}、Shimon SHOJI\eauthorrefmark{1}、Yamato MATSUI\eauthorrefmark{1}、Masamitsu YAMAZAKI\eauthorrefmark{1}、Leon TONOUCHI\eauthorrefmark{1}、\\Yasuo HAYASHIBARA\eauthorrefmark{1}、Ryuichi UEDA\eauthorrefmark{1}}

\affiliation{千葉工業大学 未来ロボティクス学科 Cat Aチーム/Bチーム}

\begin{document}
\maketitle

\authorreftext{1}{千葉工業大学先進工学部未来ロボティクス学科}
\authorreftext{2}{千葉工業大学院先進工学研究科未来ロボティクス専攻}
\eauthorreftext{1}{Department of Advanced Robotics、Faculty of Advanced Engineering、Chiba Institute of Technology}
\eauthorreftext{2}{Department of Advanced Robotics、Graduate School of Advanced Engineering、Chiba Institute of Technology}

% 本文
\section{緒言}

千葉工業大学未来ロボティクス学科Cat Aチーム/Bチームは、
図\ref{fig:raspicat}に見られる小型で簡素なセンサ構成のロボットを、
屋外で安定して自律走行させるためのソフトウェアを研究している.
簡素さをテーマにするのは、
安価な屋外移動ロボットのシステムの実現と、
情報が不確かなときのロボットの巧妙な振る舞いの研究に興味があるからである。
最新、高性能なセンサを次々に導入する方針だと、
アルゴリズムの工夫にかける時間や動機が減り、
また、ロボットの計算機、台車、電池の大型化、
製作、メンテナンス費用の増大を招く。
そこで、あえて計算負荷の低いセンサと
小型の計算機での安定した自己位置推定やナビゲーション
の実現に取り組んでいる。


本年度は、
\begin{itemize}
\item 計算機としてRaspberry Pi 4 Model Bのみを使用(Aチーム: DRAMが8GBのモデル、Bチーム: DRAMが4GBのモデル)
\item 外界センサとして2次元LiDARのみを使用
\item 移動量の算出に内界センサ(IMUや車輪のロータリーエンコーダ)を不使用
\end{itemize}
という構成で、つくばチャレンジ本走行での完走を試みた。
昨年度はノート型PC(Core i7-10750H)を計算機として使用したが、
本年度は用いず、ナビゲーションに関する全ての計算を
Raspberry Pi 4ひとつで実現した。
ロボットはアールティ社製の既製品、Raspberry Pi Catである。



\begin{figure}[h]
 	\begin{center}
 		\includegraphics[width=1.0\linewidth]{figs/raspicat.pdf}
 		\caption{Raspberry Pi Cat(左: Aチーム、右: Bチーム)}
 		\label{fig:raspicat}
 	\end{center}
\end{figure}

本年度の本走行の結果を表\ref{MainRun}に示す。
記録としては両チームとも確認走行区間の終わりか、少し超えた程度の
地点までしか到達できなかったが、
Bチームについては、ノートPCを計算機に用いた昨年の成績を上回った。
また、実験走行では, Aチームは4回の人の介入のみでゴールまで到達することができ、
完走に向けて目処が立った。

\begin{table}[h]
  \caption{各チームの本走行の結果}
  \label{MainRun}
	\begin{tabular}{|c|c|p{5.4cm}|}
    \hline
	チーム & 走行距離 & リタイアの理由 \\
    \hline
	A & 200[m] & 市役所前の横断歩道の横断直前で,点字ブロックに車輪を取られ自己位置推定が破綻\\
    \hline
	B & 369[m] & 2つめの一時停止地点において一時停止の規則に違反 \\ 
    \hline
  \end{tabular}
\end{table}


本稿では、本チームのような2次元LiDAR、Raspberry Pi
という構成での屋外移動ロボットの構築方法を説明する。
また、この構成で特に重要となる自己位置推定について、
つくばチャレンジで起こったことを踏まえて議論する。
本稿の構成は次の通りである. 
2章ではリタイヤの原因について、
3章ではロボットのシステムについて説明をし、
自己位置推定やその他ナビゲーションのソフトウェアに
見つかった課題について説明をする. 
4章では結言を述べる。

%2章、3章では、参加したロボットの構成について、
%それぞれハードウェア、ソフトウェアの面から説明する. 
%4章では本年度のつくばチャレンジでのチームの活動や、
%実験走行、本走行での結果を説明する.
%5章で考察を行い、6章で得られた知見、7章で結言を述べる. 
%@@@変更あり@@@

\section{リタイアの原因}

\subsection{実験走行でのリタイア}


\section{ロボットのシステム}

\subsection{センシングから自己位置推定までのシステム}\label{sub:localization}

\subsubsection{使用したハードウェアとソフトウェア}


\subsubsection{実験走行、本走行で見つかった課題}

\paragraph{横断歩道手前の点字ブロックでの向きの急変}


\paragraph{特徴の乏しい長い直線での不安定さ}

\subsection{経路生成、衝突回避のシステム}

\subsection{電力消費の大幅削減}

\section{結言}

\section*{謝辞}

%↑上田整理。あと、著者に名前を連ねない人がいたらここに書くと良いです。

% 参考文献
% \small
\footnotesize
\begin{thebibliography}{99}
  \bibitem{ROS}
	  Morgan Quigley {\it et al.}: ``ROS: an open-source Robot Operating System,'' 
Open-Source Software workshop of the International Conference on Robotics and Automation、2009. 

\bibitem{fox2003}
Dieter Fox:
``Adapting the Sample Size in Particle Filters Through KLD-Sampling,''
International Journal of Robotics Research、Vol. 22、No. 12、pp. 985-1003、2003. 

\bibitem{gutmann2002}
Jens-Steffen Gutmann and Dieter Fox: 
``An Experimental Comparison of Localization Methods Continued,''
Proc. of the IEEE/RSJ International Conference on Intelligent Robots and Systems (IROS),pp. 454-459、2002.
  
% \bibitem{ueda2002tdp}
% 	Ryuichi Ueda {\it et al.}: 
% ``Team description of Team ARAIBO,'' 
% Proc. of 2002 International RoboCup Symposium、2002. 

  % \bibitem{ueda2004iros}
	% Ryuichi Ueda {\it et al.}: 
  % ``Expansion Resetting for Recovery from Fatal Error in Monte Carlo Localization -- Comparison with Sensor Resetting Methods,'' Proc.of IROS,pp.2481--2486,2004.
  
  % \bibitem{map2gazebo}
  % Shiloh Curtis: ``shilohc/map2gazebo'',\url{https://github.com/shilohc/map2gazebo} (last visit: 2021-12-31).
  
  % \bibitem{move_base}
  % Eitan Marder-Eppstein: ``move\_base,'' \url{http://wiki.ros.org/move_base} (last visit: 2021-12-31).
  
  \bibitem{amcl}
  Brian Gerkey: ``amcl,'' \url{https://wiki.ros.org/amcl} (last visit: 2021-12-31).

  \bibitem{gmapping}
  Brian Gerkey: ``gmapping,'' \url{http://wiki.ros.org/gmapping} (last visit: 2021-12-31).
  
  % \bibitem{GIMP}
  % GIMP.org: ``GIMP,'' \url{https://www.gimp.org/} (last visit: 2021-12-31).
  
\bibitem{emcl2}
Ryuichi Ueda: ``ryuichiueda/emcl2,''\\\url{https://github.com/ryuichiueda/emcl2} (last visit: 2022-12-12).

\bibitem{pfc}
	Ryuichi Ueda: ``ryuichiueda/value\_iteration (amdp branch),''\\\url{https://github.com/ryuichiueda/value_iteration/tree/amdp} (last visit: 2022-12-12).

  \bibitem{raspicat}
  Ryuichi Ueda and Daisuke Sato: ``ja/raspicat,'' \url{https://wiki.ros.org/ja/raspicat} (last visit: 2021-12-31).
  
  \bibitem{youtube}
  BEIKE: ``つくばチャレンジ2022 実験走行 11/19 スタートからゴールまで自律移動(神の手4回),'' \url{https://www.youtube.com/watch?v=3gpjVhRIJDY} (last visit: 2022-12-12).
  % \bibitem{raspicat_rosbag}
  % Tatsuhiro Ikebe: ``uhobeike/raspicat\_rosbag,'' \url{https://github.com/uhobeike/raspicat_rosbag} (last visit: 2021-12-31).

  \bibitem{池邉2021}
 池邉 龍宏,曹 越,高橋 秀太,クルス ペレス アントニオ,林原 靖男,上田 隆一: 小型移動ロボットによるつくばチャレンジへの挑戦,第22回計測自動制御学会システムインテグレーション部門講演会,pp.3390-3393,2021.

% \bibitem{上田2019}
% 上田 隆一: ``詳解確率ロボティクス''、講談社、2019.

%   \bibitem{上田2020}
%  上田隆一,鈴木勇矢: 自己位置が不確かな状況における移動ロボットの危険回避行動の生成,第38回日本ロボット学会学術講演会予稿集,pp.RSJ2020AC2C2-02,オンライン開催,2020.

  % \bibitem{地図合成}
  % 川合隆太他: ``産業技術大学院大学における自律移動ロボット「産技大2号」の開発'',2019年度つくばチャレンジシンポジウム、pp.4-7、2020.

  % \bibitem{RTshop}
  % 株式会社アールティ:``Raspberry Pi Cat 屋外でも動かせる中型2輪ロボット'',
  % RT Robot Shop Products,\url{https://rt-net.jp/products/raspberry-pi-cat/} (last visit 2021-12-31)

  % \bibitem{aws2020}
	%   CIT自律ロボット研究室: ``AWSロボットデリバリーチャレンジで本研究室メンバーが優勝,'' \url{https://lab.ueda.tech/?post=20200915_aws_challenge} (last visit 2022-01-04)

  % \bibitem{学科サイト}
	%   千葉工業大学先進工学部未来ロボティクス学科: ``AWS Robot Delivery Challenge 2021 準優勝''、\url{https://www.robotics.it-chiba.ac.jp/j/?p=838} (last visit 2022-01-04)
  
  % \bibitem{つくばチャレンジロボット仕様}
  % つくばチャレンジ実行委員会事務局:``つくばチャレンジ 2021 ロボット仕様条件'',
  % \url{https://tsukubachallenge.jp/2021/regulations/specs} (last visit 2021-12-31)
  
  % \bibitem{UST-30LX}
  % 北陽電機株式会社:``UST-30LX'',\url{https://www.hokuyo-aut.co.jp/search/single.php?serial=195#spec} (last visit: 2021-12-31).
  
  % \bibitem{Turtlebot3 Burger}
  % 株式会社ロボティズ: ``Turtlebot3 Burgerの仕様'',\url{https://emanual.robotis.com/docs/en/platform/turtlebot3/features/} (last visit: 2022-1-3).
  
  % \bibitem{つくばチャレンジ公式記録}
  % つくばチャレンジ実行委員会事務局:``つくばチャレンジ2021の走行結果'',
  % \url{https://tsukubachallenge.jp/2021/records/final} (last visit 2021-12-31)

  % \bibitem{出野畑中}
	%   畑中 優一郎,出野 廣太郎,上田 隆一:``Raspberry Pi 3BのみでRaspberry Pi Catのナビゲーション(屋内環境編)'',CIT自律ロボット研究室,\url{https://lab.ueda.tech/?post=20211210} (last visit 2022-01-02)
\end{thebibliography}
\normalsize

\clearpage

%\section{付録として、各実験走行でどんなことをやっていたかを書く?}
% 当チームは、9日間分のすべての実験走行会に参加した。
% 7月2日、7月23日は確認走行区間から信号あり横断歩道手前までの自己位置推定のテストを行った。
% 地図は昨年作ったものを使い、自己位置推定のパッケージにemcl2を使った。
% また、駅・公園エリアの地図作成のためのrosbagを取った。
% 9月17日、10月1日は駅・公園エリアの走行実験を行った。
% 走行エリアが増えたことに伴い地図の容量が増大し、4GBメモリのRaspberry Piでは処理できなかった。
% メモリの容量を増やし、地図の余白部分を削ることで駅・公園エリアが走行できるようになった。
% 10月22日、10月23日は信号あり横断歩道の横断の走行テストを行った。
% 通常の走行速度では青信号中に横断しきれなかったため、
% 信号あり横断歩道の横断中のみ走行速度を変えることで、青信号中に横断しきることができた。

\end{document}
